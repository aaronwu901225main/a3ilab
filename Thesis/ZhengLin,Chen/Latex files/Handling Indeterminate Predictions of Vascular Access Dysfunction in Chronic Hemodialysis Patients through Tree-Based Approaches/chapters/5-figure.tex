\chapter{Conclusion}
\label{chapter:fig}
Taiwan, with the world's highest proportion of dialysis patients, faces significant challenges in managing vascular access dysfunction, a common complication among these patients. Early detection and accurate diagnosis are critical for reducing the risks associated with this condition and improving patient outcomes. This study proposes an indeterminate-aware tree-based machine learning framework to enhance diagnostic precision and address the inherent indeterminacies in clinical data.

Our methodology integrates Multipass Indeterminacy Estimation and an Indeterminate-Aware Data Classification approach to provide robust predictions while accounting for ambiguous cases. By introducing an extended confusion matrix and novel metrics such as leakage, overkill, indeterminacy, and imperfection, the study captures a comprehensive evaluation of model performance. These metrics not only highlight the strengths of the proposed model in reducing misclassification but also ensure cautious decision-making, particularly for borderline cases.

Our model demonstrates superior predictive performance across multiple metrics. The inclusion of indeterminate classifications further enhances diagnostic reliability, reflected in improved AUC, PPV, and reduced error rates. For example, in both AVF and AVG datasets, our approach outperformed baseline methodologies~\cite{Wu}, achieving significant improvements in accuracy, harmonic scores, and indeterminacy handling.

This research emphasizes the importance of accounting for indeterminacy in clinical decision-making. By reallocating ambiguous predictions to an "Indeterminate" category, the model avoids overly confident misclassifications, providing a balanced and cautious framework for diagnosis. The proposed approach not only improves prediction results but also aligns with the clinical need for reliable, interpretable, and actionable decision support systems.

In conclusion, the study demonstrates that integrating machine learning with indeterminacy analysis offers a powerful tool for managing complex clinical datasets. The proposed methodology has the potential to enhance patient care by enabling more accurate and informed decisions, reducing unnecessary interventions, and ensuring timely treatment for dialysis patients. This framework can serve as a foundation for future advancements in medical AI, particularly in areas requiring precise and cautious diagnostic strategies.