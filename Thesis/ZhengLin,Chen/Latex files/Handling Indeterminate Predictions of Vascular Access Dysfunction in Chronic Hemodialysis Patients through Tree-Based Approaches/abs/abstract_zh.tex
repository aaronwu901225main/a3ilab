\begin{abstractzh}

血管通路功能不全是透析患者常見且嚴重的併發症,尤其在透析患者比例全球最高的台灣,早期診斷和管理顯得尤為重要。傳統的診斷方法,如定期監測及固定血流量閾值的應用,難以找出潛在需要執行手術的病患。為解決此問題,本研究提出了一個具備不確定感知能力的樹狀機器學習框架,旨在改善血管通路功能不全的判斷方法。

該框架結合了Decision Trees、Random Forests及XGBoost等樹狀模型,並融入了先進的不確定性量化技術。通過多次擾動的方式模擬樣本的變化,實現了對不確定樣本的識別。本研究還引入了擴展混淆矩陣及創新的不確定性指標,從多角度對模型性能進行全面評估。從實驗與驗證結果表明,該框架相較於傳統方法(如KDOQI guidelines)具有更高的預測準確性和敏感性。將模棱兩可的樣本分類為不確定樣本的能力提升了臨床決策的可靠性,避免了過於自信的誤判並減少了不必要的干預。本研究在醫院的真實資料集中進行模擬,驗證所提出之方法的性能。

\vspace{5cm}

關鍵詞: 血管通路功能不全、樹模型、機器學習、不確定性量化

\end{abstractzh}