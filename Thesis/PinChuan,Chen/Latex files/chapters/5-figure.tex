\chapter{Conclusion}
In this paper, we proposed an innovative concept of a compositional conditional diffusion model to generate unseen defect component. We in a method based on class-image correlation to regulate the diffusion model for image generation. The outcomes of this research can be applied in the context of defect detection in electronic industry production lines. When encountering new defective components in the future, the compositional conditional diffusion model can be employed to generate visual representations of these components. The produced images can then be incorporated into the defect detection model for further training.

This research project aims to investigate the practical aspects of human composite cognitive abilities, emphasizing the process of learning from existing composite concepts and applying them to novel composite concepts. Specifically, in the domain of image generation, the goal is to achieve "composite zero-shot image generation and selection." Within the academic realm of artificial intelligence, the study delves into the generalization capabilities of neural networks in the context of composite zero-shot learning and generation. We look forward to further exploring the compositional condition diffusion model in a wider variety of settings and data modalities.
\label{chapter:fig}
