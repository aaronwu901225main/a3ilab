\begin{abstract}%

Vascular access dysfunction is a common and serious complication in hemodialysis patients, particularly in Taiwan, which has the highest proportion of dialysis patients globally. Early diagnosis and management are critically important. Traditional diagnostic methods, such as routine surveillance and fixed blood flow thresholds, often fail to identify patients who may require surgical intervention. To address this issue, this study proposes an indeterminate-aware, tree-based machine learning framework aimed at improving the assessment of vascular access dysfunction.

The framework integrates tree-based models, including Decision Trees, Random Forests, and XGBoost, with advanced uncertainty quantification techniques. By employing a multipass perturbation approach to simulate sample variability, the framework effectively identifies indeterminate cases. Additionally, this study introduces an extended confusion matrix and innovative uncertainty metrics to comprehensively evaluate model performance from multiple perspectives. Experimental and validation results demonstrate that the proposed framework outperforms traditional methods, such as KDOQI guidelines, in terms of predictive accuracy and sensitivity. The ability to classify ambiguous samples as indeterminate enhances the reliability of clinical decision-making, avoiding overconfident misclassifications and reducing unnecessary interventions. The method’s performance has been validated using real-world hospital datasets, confirming its efficacy and robustness.

\vspace{1cm}

Keywords: Vascular Access Dysfunction, Tree-Based Models, Machine Learning, Indeterminacy Quantification

\end{abstract}