\begin{acknowledgement}%
\thispagestyle{empty}
畢業論文完成之際,也標誌著我學生生涯的一個階段暫時告一段落。回顧這幾年的碩士生涯,從意外進入交大AI學院、涉足電腦視覺領域、日夜加強程式設計技能、獨立承擔Phison產學計畫,到最終完成論文,這一路走來,我感激許多人的指導和幫助。

首先,我必須感謝我的指導教授,馬清文教授。馬教授對研究的要求極為嚴格,常教導我們要追求完美。每當我遇到研究瓶頸或困難時,教授總會耐心討論、提供實用建議並鼓勵我不要放棄。合理的要求叫訓練,不合理的要求叫磨練,謝謝馬老師三年來的磨練。正如老師所言,這段旅程不僅拓寬了我的視野,也讓我對人生有了全新的理解。

我也要感謝實驗室的同學,但大家都樂於互助,尤其要特別感謝世倫。我本科來自護理背景,對資訊工程知識有限,是世倫快速地帶我進入狀況,成為我碩士生涯中的AI導師。同時,感謝伯宏在這段學習旅程中的陪伴。

在EDA實驗室,我認識了三位學長:弘運、博群與宙澄。特別是弘運學長的鼓勵,使我能夠一步步完成碩士論文;他的經驗讓我能快速完成寫作。他們畢業後,弘運、博群與宙澄學長也常關心我的狀況,讓我感受到許多溫暖。

最後,我要感謝我的母親。作為單親家庭的母親,撫養我成長肯定充滿挑戰。儘管我並非一個讓你完全無憂的孩子,但你對我的學業與未來始終給予無條件的支持。當聽聞我考入交大AI學院時,你為我感到高興並全力支持,讓我無需擔心學業與經濟上的壓力。感謝你無私的栽培,讓我能達到今天的成就。
\end{acknowledgement}