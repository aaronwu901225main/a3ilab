\begin{abstractzh}%

在一般的機器學習任務中,測試資料集通常和訓練資料集有同樣的分布,在圖像生成領域也是如此,生成模型往往只能產生跟訓練資料集具有同樣分布的資料,我們透過一個很簡單的例子,利用不同的方法去實現 unseen data generation,在現有的方式沒辦法做到的情況下,我們將 unseen data generation 進一步拆解成 unseen compositional image generation,並利用組合式標籤來引導生成模型。不同與以往只使用一種類別標籤來引導生成模型,我們的模型使用多種類別標籤來控制生成過程,像是屬性類別,物件類別等等,透過模型訓練學習到不同種類標籤的特徵,來實現組合式的零樣本圖像生成。

關於零樣本生成,大語言模型,如 GPT-4,與大型文生圖模型,如 DALLE-2,也具有類似零樣本生成之能力。其不同之處為,我們研究在特定領域內實踐零樣本生成之可能性與必備條件,並以中小模型來完成。為佐證組合式零樣本影像生成乃為當代機器學習技術有潛力能完成之任務,我們設計一系列由簡單到複雜的組合式零樣本生成任務。在簡單的任務中,我們提出的方法已可準確生成新組合樣本。在較為複雜的任務中,由我們的模型所生成之新組合樣本,經過挑選,亦可得到合理新樣本。

\vspace{17cm}

關鍵詞: 圖像生成、擴散模型、組合式零樣本圖像生成、組合式零樣本學習。

\end{abstractzh}